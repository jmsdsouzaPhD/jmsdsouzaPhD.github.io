\documentclass[a4paper]{book}
\usepackage[left=2cm,right=2cm,top=2cm, right=2cm]{geometry}
\usepackage[utf8]{inputenc}
\usepackage{float}
\usepackage{graphicx}

\begin{document}
	\thispagestyle{empty}
	\section*{\huge \center \includegraphics[width=.1\linewidth]{py_logo}\\Linguagem Python\\ \small  ``A melhor forma de aprender programação é programando" }
	$$ $$
		\subsection*{Introdução}
			Por se tratar de uma linguagem de \textbf{alto nível} ela é uma linguagem mais simples. Por exemplo, diferente das linguagens C e C++, o python não precisa usar  os símbolos ``;" ``\{", e ``\}" em seus códigos. A extenção para códigos escritos em python é o ``\textbf{.py}". Você pode escrever os seus códigos dentro de plataformas como \textbf{Kite} ou \textbf{Visual Studio} (há também o \textbf{Python GUI} se você trabalha com Windows), mas o mais simples é escrever os seus códigos com qualquer editor de textos que você possua (por exemplo o \textbf{gedit}) e compilar seus programas no terminal da sua máquina. Se você usa o \textbf{Linux} a sua máquina já tem o python instalado, basta rodar no seu terminal:\\
			$>>$  \textbf{python}\\
			e seu terminal irá permitir uma programação livre (a cada ENTER sua linha de comando será executada).
			
			
		\subsection*{1. Comandos Básicos}
			\begin{itemize}
				\item[\textbf{print()}] Imprimir algo na tela;
				\item[\textbf{if():}] Se o argumento for \textbf{True} (ou 1) o comando seguinte será executado, e caso o argumento for \textbf{False} (ou 0) o comando seguinte não será executado;
				\item[\textbf{while():}] executa repetidamente o comando seguinte caso o argumento for \textbf{True} (ou 1); caso contrário (argumento \textbf{False} ou 0) o comando seguinte não será executado;
			\end{itemize}
			
		\subsection*{2. Tipos de Variáveis}
			\subsubsection*{2.1. String ``texto" (str)}
			
			\subsubsection*{2.2. Inteiro (int)}
			
			\subsubsection*{2.3. Ponto Flutuante (float)}
			
			\subsubsection*{2.4. Booleano (bool)}
				Variável do tipo \textit{verdadeiro} ou \textit{falso} (\textbf{True}, \textbf{False}).
		\begin{table}[H]
			\centering
			\begin{tabular}{|c|c|}
				\hline
				Tipos de Variáveis & Símbolos \\ \hline
				string & \%s \\
				integer & \%d\\
				float & \%f  \\
				float (1 digito significativo) & \%.1f \\
				float (2 digitos significativos) & \%.2f \\
				float (3 digitos significativos) & \%.3f \\
				float (4 digitos significativos) & \%.4f \\
				float (5 digitos significativos) & \%.5f \\
				$\vdots$ & $\vdots$ \\ \hline
			\end{tabular}
		\end{table}
		
		\subsection*{3. Operação com Variáveis}
			
		\subsection*{4. Listas, Vetores e Matrizes}
		
		\subsection*{5. Comando \Large \underline{for}}
		
		\subsection*{6. Biblioteca \underline{Numpy} (Funções Matemáticas)}
		
		\subsection*{7. Biblioteca \underline{Matplotlib} (Geração de Gráficos)}
		
		\subsection*{8. Manipulando e Criando Arquivos}
		
		\subsection*{9. Criação de Funções}
		
		\subsection*{10. Biblioteca \underline{Scipy} (Funções Científicas)}
\end{document}